\documentclass[a4paper,10pt]{article}
\usepackage[utf8]{inputenc}

%opening
\title{Software development toolbox course//Notes}
\author{Anastasia Kruchinina}

\begin{document}

\maketitle

\begin{abstract}
Compiling Ergo with CMake. Usage of testing tools as CTest and CDash, or maybe something else.
``CMake is a cross-platform build systems generator which makes
it easier to build software in a unified manner on a broad set
of platforms''.


CMake supports out-of-source build.
Even the most simple project should never mix-up sources with
generated files. The reason is that it just looks better when all complicated cmake files are hiden in some subdirectory, it s easy to clean code from build files, it allows you to have more build trees for the same source.
Create for example build directory and run cmake from it, all cmake files will be located in build directory.
A good practice is to set a guard (cmake file) which prevent in-source build.




\end{abstract}

\section{Introduction to CMake}

There are not so many good tutorial about CMake. I am using excellect slides of Eric NOULARD from github (https://github.com/TheErk/CMake-tutorial).


\section{Diffuculties}


\section{Autotools vs. CMake}


\end{document}
